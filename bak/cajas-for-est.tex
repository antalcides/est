\newtcbox{\mylib}{enhanced,nobeforeafter,tcbox raise base,boxrule=0.4pt,top=0mm,bottom=0mm,
  right=0mm,left=4mm,arc=1pt,boxsep=2pt,before upper={\vphantom{dlg}},
  colframe=green!50!black,coltext=green!25!black,colback=green!10!white,
  overlay={\begin{tcbclipinterior}\fill[green!75!blue!50!white] (frame.south west)
    rectangle node[text=white,font=\sffamily\bfseries\tiny,rotate=90] {LIB} ([xshift=4mm]frame.north west);\end{tcbclipinterior}}}

\robustify{\mylib}

\pdfstringdefDisableCommands{%
  \def\mylib#1{'#1'}%
}
nota:
\newtcolorbox{marker}[1][]{enhanced,
  before skip=2mm,after skip=3mm,
  boxrule=0.4pt,left=5mm,right=2mm,top=1mm,bottom=1mm,
  colback=yellow!50,
  colframe=yellow!20!black,
  sharp corners,rounded corners=southeast,arc is angular,arc=3mm,
  underlay={%
    \path[fill=tcbcol@back!80!black] ([yshift=3mm]interior.south east)--++(-0.4,-0.1)--++(0.1,-0.2);
    \path[draw=tcbcol@frame,shorten <=-0.05mm,shorten >=-0.05mm] ([yshift=3mm]interior.south east)--++(-0.4,-0.1)--++(0.1,-0.2);
    \path[fill=yellow!50!black,draw=none] (interior.south west) rectangle node[white]{\Huge\bfseries !} ([xshift=4mm]interior.north west);
    },
  drop fuzzy shadow,#1}
  %%%%%%%%%%%%%%%%%%%%%%%%%%%%%%%%%%%%%%%%%%%%%%%5
  \newtcbexternalizetcolorbox{exdispExample}{dispExample}{environment with percent=false,minipage}{beforeafter example}

\newtcbexternalizetcolorbox{exdispExample*}{dispExample*}{environment with percent=false,minipage}{beforeafter example}

%%%------------------------------------------------------
  %%%%--------- cajas diseñadas por Walter Mora wmora2@gmail.com
  %---------------------------------------------------------------------------------
%  Entornos:  Ejemplo, teorema, proposición, lema, lista de ejercicios, 
%             caja interludio, caja simple  CON PAQUETE TCOLORBOX
%---------------------------------------------------------------------------------

%  Cajas con el paquete  tcolorbox 
%  CONTADORES: ejemplo, definicion, lema, teorema, corolario, proposicion,ejercicio 
\newcounter{tcbteo}[section]
\renewcommand{\thetcbteo}{\thesection.\arabic{tcbteo}}

\newcounter{tcbdefi}[section]
\renewcommand{\thetcbdefi}{\thesection.\arabic{tcbdefi}}

\newcounter{tcblema}[section]
\renewcommand{\thetcblema}{\thesection.\arabic{tcblema}}

\newcounter{tcbcoro}[section]
\renewcommand{\thetcbcoro}{\thesection.\arabic{tcbcoro}}

%  \newcounter{tcbListaEjercicios}[section]
%  \renewcommand{\thetcbListaEjercicios}{\thesection.\arabic{tcbListaEjercicios}}

\newcounter{tcbpropo}[section]
\renewcommand{\thetcbpropo}{\thesection.\arabic{tcbpropo}}

\newcounter{tcbvoca}[section]
\renewcommand{\thetcbvoca}{\thesection.\arabic{tcbvoca}}
%%%%%%%%%%% cajas transparentes------------
\tikzfading[name=fade up,top color=transparent!0, bottom color=transparent!80]

\tikzfading[name=fade right,
  left color=transparent!0, right color=transparent!80]


%%%%%%%%%%%%------------------------------------

\newlength{\examlen}
\tikzset{
    wnodeTeorema/.style={%
         rectangle,  top color=gray!5, bottom color=gray!5, path fading=fade right,
         inner sep=1mm,anchor=west,font=\small\bf\sffamily},
   wnodeminimo/.style={%
         rectangle,  top color=white, bottom color=white,path fading=fade right,
         text=azulF,inner sep=1mm,anchor=west,font=\small\bf\sffamily}      
}



%\begin{teorema}  o \begin{teorema}[de tal] o \begin{teorema}[][ref]
% Teorema -----------------------------------------------------
\newtcolorbox{wwteo}[3][]{%
arc=0mm,breakable,enhanced,colback=gray!5,boxrule=0pt,top=7mm,enhanced jigsaw,opacityback=0.35,
fontupper=\normalsize,step and label={tcbpropo}{#3},
overlay unbroken ={
\draw[color=colordominante,line width=0.2pt] (frame.north west)--([xshift=0pt]frame.north east);
%Caja de Título: propo --
\node[wnodeTeorema](titulopropo) at ([xshift=0pt, yshift=-4mm]frame.north west)
{\textbf{\color{colordominante} Teorema \thetcbteo\;#2}};
%Borde superior --
\draw[colordominante,line width=2.5cm] ([xshift=1.25cm, yshift=0cm]frame.north west)-- +(\examlen,3pt);
}, %overlay
overlay first ={
\draw[color=colordominante,line width=0.2pt] (frame.north west)--([xshift=0pt]frame.north east);
%Caja de Título: propo --
\node[wnodeTeorema](titulopropo) at ([xshift=0pt, yshift=-4mm]frame.north west)
{\textbf{\color{colordominante} Teorema \thetcbteo\;#2}};
%Borde superior --
\draw[colordominante,line width=2.5cm] ([xshift=1.25cm, yshift=0cm]frame.north west)-- +(\examlen,3pt);
}, %overlay
% Mantener borde en cambio de página 
overlay last ={\draw[color=colordominante,line width=0.2pt] (frame.north west)--([xshift=0pt]frame.north east);
        },
#1}
%-
\NewDocumentEnvironment{teorema}{O{} O{} O{}}{\smallskip\begin{wwteo}{#1}{#2}%
 #3}{\end{wwteo}\smallskip }
 
% TEOREMA---------------------------------------------------------
\newcommand*{\theorembreak}{\usebeamertemplate{theorem end}\framebreak\usebeamertemplate{theorem begin}}
% TEOREMA---------------------------------------------------------



%\begin{proposicion}  o \begin{proposicion}[de tal] o \begin{proposicion}[][ref]
% Proposición-----------------------------------------------------
\newtcolorbox{wwpropo}[3][]{%
arc=0mm,breakable,enhanced,colback=gray!5,boxrule=0pt,top=7mm,enhanced jigsaw,opacityback=0.35,
fontupper=\normalsize,step and label={tcbpropo}{#3},
overlay unbroken ={
\draw[color=colordominante,line width=0.2pt] (frame.north west)--([xshift=0pt]frame.north east);
%Caja de Título: propo --
\node[wnodeTeorema](titulopropo) at ([xshift=0pt, yshift=-4mm]frame.north west)
{\textbf{\color{colordominante} Proposición \thetcbpropo\;#2}};
%Borde superior --
\draw[colordominante,line width=2.5cm] ([xshift=1.25cm, yshift=0cm]frame.north west)-- +(\examlen,3pt);
}, %overlay
overlay first ={
\draw[color=colordominante,line width=0.2pt] (frame.north west)--([xshift=0pt]frame.north east);
%Caja de Título: propo --
\node[wnodeTeorema](titulopropo) at ([xshift=0pt, yshift=-4mm]frame.north west)
{\textbf{\color{colordominante} Proposición \thetcbpropo\;#2}};
%Borde superior --
\draw[colordominante,line width=2.5cm] ([xshift=1.25cm, yshift=0cm]frame.north west)-- +(\examlen,3pt);
}, %overlay
% Mantener borde en cambio de página 
overlay last ={\draw[color=colordominante,line width=0.2pt] (frame.north west)--([xshift=0pt]frame.north east);
        },
#1}
%-
\NewDocumentEnvironment{proposicion}{O{} O{} O{}}{\smallskip\begin{wwpropo}{#1}{#2}%
 #3}{\end{wwpropo}\smallskip }
% ---------------------------------------------------------



% LEMA -----------------------------------------------------------
 \newtcolorbox{wwlema}[3][]{%
arc=0mm,breakable,enhanced,colback=gray!5,boxrule=0pt,enhanced jigsaw,opacityback=0.35,
top=1mm, left=3pt,
step and label={tcblema}{#3},
fontupper={\small\bf\sffamily {\color{azulF}Lema \thetcblema \;#2}}~\normalfont, %"Lema..."+texto del cuerpo
overlay unbroken  = {%barra vertical
\draw[color=colordominante,line width=3pt] ([xshift=2pt] frame.north west)--([xshift=2pt] frame.south west);              
         },%
overlay first  = {%barra vertical
\draw[color=colordominante,line width=3pt] ([xshift=2pt] frame.north west)--([xshift=2pt] frame.south west);              
         },%
% Mantener borde en cambio de página     
overlay last ={\draw[color=gray,line width=3pt] ([xshift=2pt] frame.north west)--([xshift=2pt] frame.south west);   },    
#1}
%-
\NewDocumentEnvironment{lema}{O{} O{} O{}}{\smallskip\begin{wwlema}{#1}{#2}%
#3}{\end{wwlema}\smallskip }
%LEMA--------------------------------------------------------------


% % Corolario -------------------------------------------------------
\newtcolorbox{wwcoro}[2][]{%
arc=0mm,breakable,enhanced,colback=gray!5,boxrule=0pt,enhanced jigsaw,opacityback=0.35,
top=1mm,left=3pt,
fontupper={\small\bf\sffamily {\color{azulF}Corolario \thetcbcoro}\;}~\normalfont, 
step and label={tcbcoro}{#2},
overlay unbroken  = {%barra vertical
\draw[color=colordominante,line width=3pt] ([xshift=2pt] frame.north west)--([xshift=2pt] frame.south west);               
        },%
overlay first  = {%barra vertical
\draw[color=colordominante,line width=3pt] ([xshift=2pt] frame.north west)--([xshift=2pt] frame.south west);               
        },%
% Mantener borde en cambio de página     
overlay last ={\draw[color=gray,line width=3pt] 
                     ([xshift=2pt] frame.north west)--([xshift=2pt] frame.south west);
              }     
#1}
%-
\NewDocumentEnvironment{corolario}{O{} O{}}{\smallskip\begin{wwcoro}{#1}%
}{\end{wwcoro}\smallskip }
% Corolario------------------------------------------------------


% % Definición---------------------------------------------------
\newtcolorbox{wwdefinicion}[3][]{%
arc=0mm,breakable,enhanced,colback=gray!5,boxrule=0pt,enhanced jigsaw,opacityback=0.35,
top=6mm,fontupper=\normalsize,step and label={tcbdefi}{#3},
overlay unbroken  = {
%borde
\draw[color=gray!20,line width=0.2pt] (frame.north west)
  --([xshift=0pt]frame.north east)
  --([xshift=0pt]frame.south east);
%  --([xshift=0pt]frame.south west)--(frame.north west);
%barra vertical
\draw[color=colordominante,line width=3pt] ([xshift=2pt] frame.north west)--([xshift=2pt] frame.south west);          
%Caja de Título: defi --
\node[wnodeTeorema](titulodefi) at ([xshift=4.5pt, yshift=-3mm]frame.north west)
{\textbf{Definición \thetcbdefi \;#2}};
                }, %overlay
overlay first  = {
%borde
\draw[color=gray!20,line width=0.2pt] (frame.north west)
  --([xshift=0pt]frame.north east)
  --([xshift=0pt]frame.south east);
%  --([xshift=0pt]frame.south west)--(frame.north west);
%barra vertical
\draw[color=colordominante,line width=3pt] ([xshift=2pt] frame.north west)--([xshift=2pt] frame.south west);          
%Caja de Título: defi --
\node[wnodeTeorema](titulodefi) at ([xshift=4.5pt, yshift=-3mm]frame.north west)
{\textbf{Definición \thetcbdefi \;#2}};
                }, %overlay
% Mantener borde en cambio de página
overlay last    = {
%borde
\draw[color=gray!20,line width=0.2pt] (frame.north west)
  --([xshift=0pt]frame.north east)
  --([xshift=0pt]frame.south east);
%  --([xshift=0pt]frame.south west)--(frame.north west);
%barra vertical
\draw[color=colordominante,line width=3pt] ([xshift=2pt] frame.north west)--([xshift=2pt] frame.south west);}
#1}
%-
\NewDocumentEnvironment{definicion}{O{} O{} O{}}{\smallskip\begin{wwdefinicion}{#1}{#2}%
 #3}{\end{wwdefinicion}\smallskip }
% %DEFINICION---------------------------------------------------------


% Caja para Ejemplo --------------------------------------------------
\newcounter{tcbejem}[section]
\renewcommand{\thetcbejem}{\thesection.\arabic{tcbejem}}
\colorlet{colorfondoejemplo}{gray!5}
\definecolor{colorejemplo}{rgb}{.0,.0,.3}
% Ejemplo
\newtcolorbox{wwejemplo}[3][]{%
arc=0mm,
breakable,%drop fuzzy shadow,
enhanced,enhanced jigsaw,opacityback=0.35,
colback=colorfondoejemplo,
boxrule=0pt,
top=8mm, %Separación vertical - inicia texto
enlarge top by=\baselineskip/2+1mm,
enlarge top at break by=0mm,pad at break=2mm,
fontupper=\normalsize,
step and label={tcbejem}{#3},
overlay unbroken = {%
%barra vertical
\draw[color=azulua,line width=3pt] ([xshift=2pt] frame.north west)--([xshift=2pt] frame.south west);           
% Caja de imagen Ejemplo
\node[rectangle, 
         text=black, 
         inner sep=1mm,anchor=west,font=\small\bf\sffamily] at ([xshift=-14.3pt,yshift=-4.1mm]frame.north west)%
{\includegraphics{logos/nodoejemploC.pdf}};
% Caja numeración y descripción
\node[rectangle, 
 text=black, 
 inner sep=1mm,
 anchor=west,
 font=\normalsize] at ([xshift=1.1cm,yshift=-2.9mm]frame.north west)%
 {\fhvb{10}{\;\thetcbejem \hspace{5pt}\;\;\;#2}};
     }, % overlay
overlay first  = {%
%barra vertical
\draw[color=azulua,line width=3pt] ([xshift=2pt] frame.north west)--([xshift=2pt] frame.south west);           
% Caja de imagen Ejemplo
\node[rectangle, 
         text=black, 
         inner sep=1mm,anchor=west,font=\small\bf\sffamily] at ([xshift=-14.3pt,yshift=-4.1mm]frame.north west)%
{\includegraphics{logos/nodoejemploC.pdf}};
% Caja numeración y descripción
\node[rectangle, 
 text=black, 
 inner sep=1mm,
 anchor=west,
 font=\normalsize] at ([xshift=1.1cm,yshift=-2.9mm]frame.north west)%
 {\fhvb{10}{\;\thetcbejem \hspace{5pt}\;\;\;#2}};
     }, % overlay
%Borde cambio de páginas
overlay middle={\draw[color=azulua,line width=3pt] ([xshift=3pt] frame.north west)--([xshift=2pt] frame.south west);},
overlay last={\draw[color=gray,line width=3pt] ([xshift=3pt] frame.north west)--([xshift=2pt] frame.south west);}
#1}
%-
\NewDocumentEnvironment{ejemplo}{O{} O{} O{}}{\smallskip\begin{wwejemplo}{#1}{#2}%
 #3}{\end{wwejemplo}\smallskip }
%EJEMPLO-----------------------------------------------------------------



% CAJA de comentario-----------------------------------------------------
\definecolor{colrnodocaja}{RGB}{44,91,144}
\definecolor{colrfondocaja}{RGB}{241,241,227}
%CAJA de comentario ------------------------------------------------------
\newtcolorbox{wwcaja}[2][]{%
arc=0mm,breakable,%drop fuzzy shadow,
enhanced,colback=gray!4,enhanced jigsaw,opacityback=0.5,
boxrule=0pt,
top=3mm, %Separación vertical - inicia texto
enlarge top by=\baselineskip/2+1mm,
enlarge top at break by=0mm,pad at break=2mm,
fontupper=\normalsize,
%step and label={tcbca}{#3},
%Borde
overlay unbroken={\draw[color=gray!2,line width=0.2pt] (frame.north west)
  --([xshift=0pt]frame.north east)
  --([xshift=0pt]frame.south east)
  --([xshift=0pt]frame.south west)--(frame.north west);
% Caja de Título CAJA
\node[ rectangle, %minimum width=0cm, minimum height=0.0cm,
         top color=azulua, bottom color=azulua,
         inner sep=0.5mm,anchor=west,font=\normalsize]at ([xshift=-0.5pt,  yshift=2.30mm]frame.north west){\fhvb{10}{\color{white} #2}};
         },
%Borde
overlay first={\draw[color=gray!2,line width=0.2pt] (frame.north west)
  --([xshift=0pt]frame.north east)
  --([xshift=0pt]frame.south east)
  --([xshift=0pt]frame.south west)--(frame.north west);
% Caja de Título CAJA
\node[ rectangle, %minimum width=0cm, minimum height=0.0cm,
         top color=azulua, bottom color=azulua,
         inner sep=0.5mm,anchor=west,font=\normalsize]at ([xshift=-0.5pt,  yshift=2.30mm]frame.north west){\fhvb{10}{\color{white} #2}};
         },
%Borde cambio de página
overlay last={\draw[color=gray!2,line width=0.2pt] (frame.north west)
  --([xshift=0pt]frame.north east)
  --([xshift=0pt]frame.south east)
  --([xshift=0pt]frame.south west)--(frame.north west);}
#1}
%-
\NewDocumentEnvironment{caja}{O{} O{}}{\smallskip\begin{wwcaja}{#1}%
 #2}{\end{wwcaja}\smallskip }
% CAJA de comentario

%CAJA simple---------------------------------------------------------------------
\newtcolorbox{wwbox}[1][]{%
arc=0mm,breakable,drop fuzzy shadow,
enhanced,colback=grisamarillo,enhanced jigsaw,opacityback=0.35,
boxrule=0pt,
top=3mm, %Separación vertical - inicia texto
enlarge top by=\baselineskip/2+1mm,
enlarge top at break by=0mm,pad at break=2mm,
fontupper=\normalsize,
%step and label={tcbca}{#3},
%Borde
overlay unbroken={\draw[color=azulua,line width=0.5pt] (frame.north west)
  --([xshift=0pt]frame.north east)
  --([xshift=0pt]frame.south east)
  --([xshift=0pt]frame.south west)--(frame.north west);
        },
%Borde
overlay first={\draw[color=azulua,line width=0.5pt] (frame.north west)
  --([xshift=0pt]frame.north east)
  --([xshift=0pt]frame.south east)
  --([xshift=0pt]frame.south west)--(frame.north west);
        },
%Borde cambio de página
overlay last={\draw[color=azulua,line width=0.5pt] (frame.north west)
  --([xshift=0pt]frame.north east)
  --([xshift=0pt]frame.south east)
  --([xshift=0pt]frame.south west)--(frame.north west);}
#1}

 \newenvironment{scaja}[1][]{\bigskip\begin{wwbox}%
 #1}{\end{wwbox}}	
% Fin CAJA simple

%CAJA vocabulario----------------------------------------------------------------
 \newtcolorbox{vocabox}[3][]{%
arc=0mm,breakable,enhanced,colback=white,boxrule=0pt,
top=1mm, left=3pt,enhanced jigsaw,opacityback=0.35,
step and label={tcbvoca}{#3},
fontupper={\small\bf\sffamily {\color{azulF}Vocabulario \thetcbvoca \;#2}}~\normalfont, %"Lema..."+texto del cuerpo
overlay unbroken  = {%barra vertical
\draw[color=white,line width=3pt] ([xshift=2pt] frame.north west)--([xshift=2pt] frame.south west);             
         },%
overlay first  = {%barra vertical
\draw[color=white,line width=3pt] ([xshift=2pt] frame.north west)--([xshift=2pt] frame.south west);             
         },%
% Mantener borde en cambio de página     
overlay last ={\draw[color=white,line width=3pt] ([xshift=2pt] frame.north west)--([xshift=2pt] frame.south west);   },    
#1}
%-
\NewDocumentEnvironment{vocabulario}{O{} O{} O{}}{\smallskip\begin{vocabox}{#1}{#2}%
#3}{\end{vocabox}\smallskip }
	
% Fin vocabulario
%%%%%%%%%%%%%%%%%%%%%%%%%%%%%%%%%%%%%%%%%%%%%%%%%%%%%%%%%%%%%%%%%%%%%%%%%%%
%%%%%%%%%%%%%%%%%%%% Cajas/cajas con marcos %%%%%%%%%%%%%%
%%%%%%%%%%%%%%%%%%%%%%%%%%%%%%%%%%%%%%%%%%%%%%%%%%%%%%%%%%%%%%%%%%%%%%%%%%%
\newtheoremstyle{colordominantenumbox}%  nombre del estilo (theorem)
{0pt}% Espacio arriba
{0pt}% Espacio abajo
{\normalfont}% % fuente del cuerpo
{}% cantidad de identación
{\small\bf\sffamily\color{colordominante}}%  Fuente del encabezado
{\;}% Puntuacion después del encabezado del teorema
{0.25em}% Espacio después del encabezado del teorema
{\small\sffamily\color{colordominante}\thmname{#1}\nobreakspace\thmnumber{\@ifnotempty{#1}{}\@upn{#2}}% texto encabezado: (e.g. Teorema 2.1)
\thmnote{\nobreakspace\the\thm@notefont\sffamily\bfseries\color{black}---\nobreakspace#3.}} % Descripción del teorema
\renewcommand{\qedsymbol}{$\blacksquare$}% Optional qed square

\newtheoremstyle{blacknumex}% nombre del estilo (theorem)
{5pt}% Espacio arriba
{5pt}% Espacio abajo
{\normalfont}% fuente del cuerpo
{} % Indent amount
{\small\bf\sffamily}% Fuente del encabezado
{\;}% Puntuacion después del encabezado del teorema
{0.25em}% Espacio después del encabezado del teorema
{\small\sffamily{\tiny\ensuremath{\blacksquare}}\nobreakspace\thmname{#1}\nobreakspace\thmnumber{\@ifnotempty{#1}{}\@upn{#2}}% texto encabezado: (e.g. Teorema 2.1)
\thmnote{\nobreakspace\the\thm@notefont\sffamily\bfseries---\nobreakspace#3.}}% opcional: Descripción del teorema

\newtheoremstyle{blacknumbox} % Theorem style name
{0pt}% Space above
{0pt}% Space below
{\normalfont}% Body font
{}% Indent amount
{\small\bf\sffamily}% Theorem head font
{\;}% Punctuation after theorem head
{0.25em}% Space after theorem head
{\small\sffamily\thmname{#1}\nobreakspace\thmnumber{\@ifnotempty{#1}{}\@upn{#2}}% Theorem text (e.g. Theorem 2.1)
\thmnote{\nobreakspace\the\thm@notefont\sffamily\bfseries---\nobreakspace#3.}}% Optional theorem note

\newtheoremstyle{especialnumbox} % Theorem style name
{0pt}% Space above
{0pt}% Space below
{\normalfont}% Body font
{}% Indent amount
{\small\bf\sffamily}% Theorem head font
{\;}% Punctuation after theorem head
{0.25em}% Space after theorem head
{\small\sffamily\color{beamer@barcolor}\thmname{#1}\nobreakspace\thmnumber{\@ifnotempty{#1}{}\@upn{#2}}% Theorem text (e.g. Theorem 2.1)
\thmnote{\nobreakspace\the\thm@notefont\sffamily\bfseries---\nobreakspace#3.}}% Optional theorem note

%%%%%%%%%%%%%%%%%%%%%%%%%%%%%%%%%%%%%%%%%%%%%%%%%%%%%%%%%%%%%%%%%%%%%%%%%%%
%%%%%%%%%%%%% Cajas/cajas sin marcos %%%%%%%%%%%%%
%%%%%%%%%%%%%%%%%%%%%%%%%%%%%%%%%%%%%%%%%%%%%%%%%%%%%%%%%%%%%%%%%%%%%%%%%%%
\newtheoremstyle{colordominantenum}% % Theorem style name
{5pt}% Space above
{5pt}% Space below
{\normalfont}% % Body font
{}% Indent amount
{\small\bf\sffamily\color{colordominante}}% % Theorem head font
{\;}% Punctuation after theorem head
{0.25em}% Space after theorem head
{\small\sffamily\color{colordominante}\thmname{#1}\nobreakspace\thmnumber{\@ifnotempty{#1}{}\@upn{#2}}% Theorem text (e.g. Theorem 2.1)
\thmnote{\nobreakspace\the\thm@notefont\sffamily\bfseries\color{black}---\nobreakspace#3.}} % Optional theorem note
\renewcommand{\qedsymbol}{$\blacksquare$}% Optional qed square
\makeatother

% Defines the theorem text style for each type of theorem to one of the three styles above
\newcounter{dummy} 
\numberwithin{dummy}{section}
\theoremstyle{colordominantenumbox}
\newtheorem{theoremeT}[dummy]{Teorema}
\newtheorem{problema}{Problema}[section]
\newtheorem{exerciseT}{Ejercicio}[section]
\theoremstyle{blacknumex}
%\newtheorem{exampleT}{Example}[section]
\theoremstyle{blacknumbox}
%\newtheorem{vocabulario}{Vocabulario}[section]
\newtheorem{definitionT}{Definición}[section]
\newtheorem{corollaryT}[dummy]{Corolario}
\theoremstyle{especialnumbox}
\newtheorem{ejerciciosT}{{\fhvb{12}{Ejercicios}\;}}[section]

% Fin mis entornos---------------------------------------------------------------
\usepackage{listings}
%%%%% Definicion de listing===========
%% % % % % %listing
     \lstdefinestyle{Source}{
   basicstyle=\small\ttfamily\bf,language={[LaTeX]TeX}, numbersep=5mm, numbers=left, numberstyle=\tiny, % number style
   breaklines=true,framexleftmargin=10mm, xleftmargin=10mm,
   backgroundcolor=\color{azulua!20!white},frameround=fttt,escapeinside=??,
   rulecolor=\color{ptctitle},
   morekeywords={% Give key words here                                         % keywords
       maketitle},
   keywordstyle=\color[rgb]{0,0,1},                    % keywords
           commentstyle=\color[rgb]{0.133,0.545,0.133},    % comments
           stringstyle=\color[rgb]{0.627,0.126,0.941},  % strings
   %columns=fullflexible  
   extendedchars=\true,
inputencoding=utf8x }
%   % % % % % %
\usepackage{caption}
\DeclareCaptionFont{white}{\color{white}}
\DeclareCaptionFormat{listing}{\colorbox{gray!20}{\parbox{\dimexpr\textwidth-2\fboxsep\relax}{\color{white}C\'odigo \thesection .\ 
\thesource\ #3}}}
\captionsetup[source]{format=listing,labelfont=white,textfont=white, singlelinecheck=false, margin=0pt, font={bf,footnotesize}}
\newcounter{source}[section]
\lstnewenvironment{source}[2][]
{\refstepcounter{source}
\captionsetup{options=source}
\lstset{%
basicstyle=\tiny\ttfamily\bf,language={[LaTeX]TeX},caption=#1,label=#2,  
numbersep=5mm, numbers=left, numberstyle=\tiny, % number style
breaklines=true,framexleftmargin=10mm, xleftmargin=10mm,
%backgroundcolor=\color{azulua!20!white},frameround=fttt
,escapeinside=??,
rulecolor=\color{ptctitle},
morekeywords={% Give key words here                                         % keywords
    maketitle},
keywordstyle=\color[rgb]{0,0,1},                    % keywords
        commentstyle=\color[rgb]{0.133,0.545,0.133},    % comments
        stringstyle=\color[rgb]{0.627,0.126,0.941},  % strings
%columns=fullflexible 
extendedchars=\true,
inputencoding=utf8x  
}
        }
{}
%%%%%%%%%%%%%%%%%%%%%------------------------------------------------------------
\usepackage[framemethod=default]{mdframed} % Se requiere para la creación de las cajas teorema, definición, ejercicios y corolario
% Exercise box	  
\newmdenv[skipabove=7pt,
skipbelow=7pt,
rightline=false,
leftline=true,
topline=false,
bottomline=false,
backgroundcolor=colordominante!10,
linecolor=colordominante,
innerleftmargin=5pt,
innerrightmargin=5pt,
innertopmargin=5pt,
innerbottommargin=5pt,
leftmargin=0cm,
rightmargin=0cm,
linewidth=4pt]{eBox}	

% Caja de ejercicios
\newmdenv[skipabove=7pt,
skipbelow=7pt,
rightline=false,
leftline=false,
topline=false,
bottomline=false,
linecolor=gray,
backgroundcolor=black!2,
innerleftmargin=5pt,
innerrightmargin=5pt,
innertopmargin=5pt,
leftmargin=0cm,
rightmargin=0cm,
linewidth=4pt,
innerbottommargin=5pt]{ejerBox}	
		  

% Crea un ambiente para cada tipo de teorema y le asigna un estilo de texto
% de la sección "Estilos:..." 

\newenvironment{ejercicioT}{\begin{eBox}\begin{exerciseT}}{\hfill{\color{colordominante}\tiny\ensuremath{\blacksquare}}\end{exerciseT}\end{eBox}}				  % Entorno para notas 
\newenvironment{nota}{\par\vskip10pt\small % Espacio vertical blanco, por encima de la observación y fuente de menor tamaño.
\begin{list}{}{
\leftmargin=35pt % Indentación a la izquierda
\rightmargin=25pt}\item\ignorespaces % Indentación a la derecha
\makebox[-2.5pt]{\begin{tikzpicture}[overlay]
\node[draw=colordominanteF,line width=1pt,circle,fill=colordominante!25,font=\sffamily\bfseries,inner sep=2pt,outer sep=0pt] at (-15pt,0pt){\textcolor{colordominanteF}{N}};\end{tikzpicture}} % ! Orange, en el círculo
\advance\baselineskip -1pt}{\end{list}\vskip5pt} % Interlineado más apretada y un espacio en blanco después de la observación


%----------------------------------------------------------------------------------------
% Listas de ejercicios con el paquete answers
% \begin{ejercicios} --- \end{ejercicios} para listas simples
% Requiere usar un entorno amsthm porque se estableció así para estas listas arriba
%----------------------------------------------------------------------------------------
\usepackage{answers}
\newtheorem{exer}{}[section]
\newenvironment{ejer}{\begin{exer}\normalfont}{\end{exer}}
\Newassociation{solu}{Soln}{ans}

% Entorno personalizado---------------------------------------------------

%\strut es un ensanchamiento para habilitar el título "Ejercicios"
\newenvironment{ejercicios}{\begin{ejerBox}\begin{ejerciciosT}\strut\smallskip
}{\end{ejerciciosT}\end{ejerBox}}	

% Soluciones al final del documento
\def\soluciones{%
\expandafter\ifx\csname Closesolutionfile\endcsname \relax\else
\Closesolutionfile{ans}\fi
}
\def\solucionesCap#1{\section*{Soluciones de la Secci\'on #1}\input{ans#1}}

% Entorno con Caja para ejercicios-----------------------------------------
%\begin{cajaejercicios}  o \begin{cajaejercicios}[de tal] 
%                        o \begin{cajaejercicios}[][ref]
% Entorno personalizado---------------------------------------------------
\definecolor{colorejercicios}{RGB}{99,42,134}
 \definecolor{beamer@barcolor}{RGB}{21,46,128}%azul
  \definecolor{beamer@headercolor}{RGB}{226,107,59}%naranja
\newcounter{tcbejer}[section]
\renewcommand{\thetcbejer}{\thesection.\arabic{tcbejer}}

\newtcolorbox{wwejercicios}[3][]{%
enhanced jigsaw,opacityback=0.35,arc=0mm,breakable,%drop fuzzy shadow,
colback=gray!5,boxrule=0pt,top=7mm,
fontupper=\normalsize,step and label={tcbejer}{#3},
overlay unbroken = {
%Borde grueso superior
\draw[color=beamer@headercolor,line width=3pt] (frame.north west)--([xshift=0pt]frame.north east);
%Caja de Título: Ejer --
\node[rounded corners=3pt,  draw=beamer@headercolor, top color=white, bottom color=white, thick,inner sep=1mm,anchor=west, font=\small\bf\sffamily](tituloejer) at ([xshift=5mm, yshift=0mm]frame.north west)
{\textbf{\color{beamer@barcolor}  Ejercicios \thetcbejer \;#2}};
%borde línea inferior
 \draw[color=beamer@headercolor,line width=0.2pt] (frame.south west)--([xshift=0pt]frame.south east); 
},%overlay
overlay first = {
%Borde grueso superior
\draw[color=beamer@headercolor,line width=3pt] (frame.north west)--([xshift=0pt]frame.north east);
%Caja de Título: Ejer --
\node[rounded corners=3pt,  draw=beamer@headercolor, top color=white, bottom color=white, thick,inner sep=1mm,anchor=west, font=\small\bf\sffamily](tituloejer) at ([xshift=5mm, yshift=0mm]frame.north west)
{\textbf{\color{beamer@barcolor}  Ejercicios \thetcbejer \;#2}};
%borde línea inferior
 \draw[color=beamer@headercolor,line width=0.2pt] (frame.south west)--([xshift=0pt]frame.south east); 
},%overlay
% % Mantener borde en cambio de página 
% overlay middle = {\draw[color=colordominante,line width=0.2pt] (frame.north west)--([xshift=0pt]frame.north east);
%                 } 
overlay middle ={},
overlay last = { %borde línea inferior
 \draw[color=beamer@headercolor,line width=0.2pt] (frame.south west)--([xshift=0pt]frame.south east); 
                } 
#1}
%-
\NewDocumentEnvironment{cajaejercicios}{O{} O{} O{}}{\smallskip\begin{wwejercicios}{#1}{#2}%
 #3}{\end{wwejercicios}\smallskip }
% ejercicios---------------------------------------------------------
%%%%%%%%%%%%%%%%%%%%%%%%%%%%%%%%%%%%%%%%%%%%%%%%%%%%%%%%%%%%%%%%%%%%%%%%%%%%%%%%%%%%%%%%%%%%%%%%%%%%
\newcounter{tcbejerc}[section]
\renewcommand{\thetcbejerc}{\thesection.\arabic{tcbejerc}}

\newtcolorbox{wwejercicio}[3][]{%
enhanced jigsaw,opacityback=0.35,arc=0mm,breakable,%drop fuzzy shadow,
colback=gray!5,boxrule=0pt,top=7mm,
fontupper=\normalsize,step and label={tcbejerc}{#3},
overlay unbroken = {
%Borde grueso superior
\draw[color=beamer@headercolor,line width=3pt] (frame.north west)--([xshift=0pt]frame.north east);
%Caja de Título: Ejer --
\node[rounded corners=3pt,  draw=beamer@headercolor, top color=white, bottom color=white, thick,inner sep=1mm,anchor=west, font=\small\bf\sffamily](tituloejer) at ([xshift=5mm, yshift=0mm]frame.north west)
{\textbf{\color{beamer@barcolor}  Ejercicio \thetcbejerc \;#2}};
%borde línea inferior
 \draw[color=beamer@headercolor,line width=0.2pt] (frame.south west)--([xshift=0pt]frame.south east); 
},%overlay
overlay first = {
%Borde grueso superior
\draw[color=beamer@headercolor,line width=3pt] (frame.north west)--([xshift=0pt]frame.north east);
%Caja de Título: Ejer --
\node[rounded corners=3pt,  draw=beamer@headercolor, top color=white, bottom color=white, thick,inner sep=1mm,anchor=west, font=\small\bf\sffamily](tituloejer) at ([xshift=5mm, yshift=0mm]frame.north west)
{\textbf{\color{beamer@barcolor}  Ejercicio \thetcbejerc \;#2}};
%borde línea inferior
 \draw[color=beamer@headercolor,line width=0.2pt] (frame.south west)--([xshift=0pt]frame.south east); 
},%overlay
% % Mantener borde en cambio de página 
% overlay middle = {\draw[color=colordominante,line width=0.2pt] (frame.north west)--([xshift=0pt]frame.north east);
%                 } 
overlay middle ={},
overlay last = { %borde línea inferior
 \draw[color=beamer@headercolor,line width=0.2pt] (frame.south west)--([xshift=0pt]frame.south east); 
                } 
#1}
%-
\NewDocumentEnvironment{ejercicio}{O{} O{} O{}}{\smallskip\begin{wwejercicio}{#1}{#2}%
 #3}{\end{wwejercicio}\smallskip }
% ejercicios---------------------------------------------------------

% Comandos para paquete answers
% pregunta-solución
\newcommand{\exersol}[2]{
\begin{ejer} 
#1\\ \scantokens{\begin{solu}#2\end{solu}}
\end{ejer}}
% listas \item pregunta-solución
\newcommand{\itemps}[2]{\item #1\scantokens{\begin{solu}#2\end{solu}}}
%\usepackage[shortlabels]{enumitem}
\newcommand{\bex}{\scantokens{\begin{solu} \end{solu}}\begin{enumerate}[label=\alph*.)]}
\newcommand{\eex}{\end{enumerate}}
%\begin{sol} \end{sol}
%%% pagina 170 del manual,258,285, 300,311,399